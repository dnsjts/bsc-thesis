\chapter{Bevezető}


\section{Motiváció}

A téma feldolgozásakor felmerült másodlagos célként egy olyan dokumentum létrehozása, amely
a fejlesztők számára jobban értelmezhető, mint maga a \emph{Common Criteria}, hiszen ahhoz, hogy
hosszútávon is fenntartható legyen a termékek biztonsága, szoros együttműködés szükségeltetik
a terméket fejlesztőkkel is, amelynek elengedhetetlen feltétele, hogy legyen nekik is átfogó képük
a termékbiztonság kialakításáról.

\section{A feladat értelmezése}

\section{A szakdolgozat felépítése}
A második fejezetben áttekintést adok a modernizálandó rendszerről és környezetéről, bemutatom az
alkalmazott módszertan általános részeit. A harmadik fejezetben a módszertan néhány osztályán
keresztül bemutatom a Common Criteria értelmezését, és alkalmazását.
A negyedik fejezetben végül összefoglalom a feladat végeredményét, és felvázolom a szakdolgozat
továbbfejlesztési lehetőségeit.
