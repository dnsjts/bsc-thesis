\chapter{Bevezetés}

\section{Motiváció}

Szoftvert fejleszteni nehéz, a fenntarható szoftverfejlesztés pedig még nehezebb feladat.
Egy szoftverterméknél kívülről nézve fontos annak minél nagyobb funkciókészlete, belülről nézve
pedig fontos a megvalósítás mérnöki értelemben vett szépsége.
Mivel a fejlesztési idő korlátos, ezért stratégiai döntést kell hozni, hogy mikor melyik szempontot
vesszük előtérbe: tipikusan egy új termék esetén mindenáron annak funkcionalitását kell növelni
annak érdekében, hogy minél jobban elterjedjen, minél szélesebb körben lehessen használni.  Ezért
sajnos fel kell olykor áldozni a megvalósítás szépségét, és szuboptimális megoldásokkal is meg kell
elégedni. Ezt a jelenséget \emph{technológiai adósság}\cite{magnusson2014technology} felvételének is
nevezik, amely analóg a pénzügyi értelemben vett adóssággal abban az értelemben, hogy ezért az
adósságért kamatot kell fizetni a termék továbbfejlesztésekor többletmunka formájában.

Esetünkben a többletmunka annyira jelentős, hogy az a további fejlesztést jelentősen akadályozza.
Ilyenkor érdemes megállni egy pillanatra, és az adósságot törleszteni. A \emph{syslog-ng} életében
ez a pillanat a közelmúltban érkezett el.

\section{A feladat értelmezése}
Annak érdekében, hogy felszámoljuk ezt az adósságot a \emph{Common Criteria} eszköztárához nyúltunk,
amely ugyan elsődlegesen a termékbiztonságot kívánja elérni a minőségbiztosítás növelésével,
mégis használható a mi céljaink eléréséhez is, mivel képes a fejlesztés során előforduló anomáliák
szisztematikus felderítésére is.

A téma feldolgozásakor felmerült célként egy olyan dokumentum létrehozása, amely a fejlesztők
számára jobban értelmezhető, mint maga a \emph{Common Criteria}, hiszen ahhoz, hogy hosszútávon is
fenntartható legyen a termékek biztonsága, szoros együttműködés szükségeltetik a terméket
fejlesztőkkel, amelynek elengedhetetlen feltétele, hogy legyen nekik is átfogó képük
a termékbiztonság kialakításának módjáról.

\section{A szakdolgozat felépítése}
A második fejezetben áttekintést adok a modernizálandó rendszerről és környezetéről, bemutatom az
alkalmazott módszertan általános részeit. A harmadik fejezetben a módszertan néhány osztályán
keresztül bemutatom a Common Criteria értelmezését, és alkalmazását. A negyedik fejezetben végül
összefoglalom az elért eredményeket, és felvázolom a szakdolgozat továbbfejlesztési lehetőségeit.
