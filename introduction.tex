%----------------------------------------------------------------------------
\chapter{Bevezető}%\addcontentsline{toc}{chapter}{Bevezető}
%----------------------------------------------------------------------------

A bevezetõ tartalmazza a diplomaterv-kiírás elemzését, történelmi elõzményeit, a feladat
indokoltságát (a motiváció leírását), az eddigi megoldásokat, és ennek tükrében a hallgató
megoldásának összefoglalását.

A bevezető szokás szerint a diplomaterv felépítésével záródik, azaz annak rövid leírásával, hogy
melyik fejezet mivel foglalkozik.

\todo[inline]{Bevezetés megírása}
\section{Motiváció}
Legjobb tudásom szerint még nem készült korábban 

A téma feldolgozásakor felmerült másodlagos célként egy olyan dokumentum létrehozása, amely
remélhetőleg a fejlesztők számára jobban értelmezhető, mint maga a \emph{Common Criteria}, hiszen
ahhoz, hogy hosszútávon is fenntartható legyen a termékek biztonsága, szoros együttműködés
szükségeltetik a terméket fejlesztőkkel is, amelynek elengedhetetlen feltétele, hogy legyen nekik is
átfogó képük a termékbiztonság kialakításáról.
\section{A feladat értelmezése}
\todo[inline]{Feladat értelmezésének megírása a korábbi összefoglalók alapján.}

\section{A szakdolgozat felépítése}
A második fejezetben áttekintést adok a modernizálandó rendszerről és környezetéről, továbbá az
alkalmazott módszertanba betekintést engedek.
A harmadik fejezetben a módszertan néhány osztályán keresztül bemutatom a , majd a negyedik
fejezetben összefoglalom , és megemlítem a továbbfejlesztési lehetőségeket.
