\chapter{A metodologia alkalmazása}

\section{Termékfejlesztési folyamatok vizsgálata}
\subsection{Észlelt hibák biztonsági vizsgálata}

Lsd: Flaw remediation (ALC\_FLR)

\subsection{Fejlesztési- és karbantartási folyamatok}

\subsection{Configuration Management}
A \emph{Configuration Management} (továbbiakban: CM) egy olyan rendszer, amely arra hivatott,
hogy azokat a változtatásokat nyomon tudja követni, amelyek hatással lehetnek a vizsgált termékre.
Amennyiben csak jól meghatározott, autorizált változtatásokat engedélyezünk, biztosíthatjuk a
termékünk integritását.
Természetesen mielőtt egyáltalán bevezetnénk a \emph{CM}-et, feltétlenül szükséges meghatározni
annak hatókörét. Erre szolgál a \emph{ALC\_CMS} szekció.

\subsubsection{Configuration Management Scope}
\begin{itemize}
    \item{ALC\_CMS.1}
    \item{ALC\_CMS.2}
    \item{ALC\_CMS.3}
    \item{ALC\_CMS.4}
    \item{ALC\_CMS.5}
\end{itemize}

\subsubsection{Configuration Management}
\begin{itemize}
\end{itemize}

\subsection{Delivery}

\section{Development Security}

