\chapter{A metodologia alkalmazása a termékfejlesztési folyamatokra}

A \emph{Common Criteria} széleskörű követelményrendszerét olvasva könnyedén a bőség zavarával
küzdhetünk a megfelelő kezdési pont kiválasztásakor. Ez a fejezet azokat a komponenseket szedte egy
csokorba, amelyek a termékfejlesztési folyamatokhoz kapcsolódnak, méghozzá abban a sorrendben, ahogy
az a termékfejlesztés életciklusában előfordul.

\section{Felfedezett hibák vizsgálata biztonsági szempontból (ALC\_FLR)}

Az alábbi szekció a karbantartási folyamat azon részével foglalkozik, miszerint a felfedezett
hibákat (különös tekintettel a biztonsági kockázatot hordozó hibákra) hogyan is kezeljük. A kezelés
fogalmát jelenleg nem tudjuk pontosan meghatározni, ennek feloldása az egyes szinteken lesz
szükséges.

A kiértékelési komponens (ALC\_FLR osztály) szintezésére igaz, hogy a magasabb szinteken
szigorúbbak a hibakezelési folyamatok, és az irányelvek.

Mivel a három szint kezelése lényegében azonos eszközöket fog igénybe venni, ezért egy
lépésben a legmagasabbat is megvalósíthatjuk, így a köztes lépések megvalósításának tanulmányozása
nem hordozna többletértéket.

\subsection{ALC\_FLR.1 - Basic flaw remediation}
\begin{quote}
    \begin{description}
        \item[ALC\_FLR.1.1D]{The developer shall document and provide flaw remediation procedures
            addressed to TOE developers.}
        \item[ALC\_FLR.1.1C]{The flaw remediation procedures documentation shall describe the
            procedures used to track all reported security flaws in each release of the TOE.}
        \item[ALC\_FLR.1.2C]{The flaw remediation procedures shall require that a description of the
            nature and effect of each security flaw be provided, as well as the status of finding
            a correction to that flaw.}
        \item[ALC\_FLR.1.3C]{The flaw remediation procedures shall require that corrective actions
            be identified for each of the security flaws.}
        \item[ALC\_FLR.1.4C]{The flaw remediation procedures documentation shall describe the
            methods used to provide flaw information, corrections and guidance on corrective actions
            to TOE users.}
    \end{description}
\end{quote}

\todo[inline]{Célja a Basic flaw remediation-nak}

\subsection{ALC\_FLR.2 - Flaw reporting procedures}
\begin{quote}
    \begin{description}
        \item[ALC\_FLR.2.2D]{The developer shall establish a procedure for accepting and acting upon
            all reports of security flaws and requests for corrections to those flaws.}
        \item[ALC\_FLR.2.3D]{The developer shall provide flaw remediation guidance addressed to TOE
            users.}
        \item[ALC\_FLR.2.5C]{The flaw remediation procedures shall describe a means by which the
            developer receives from TOE users reports and enquiries of suspected security flaws in
        the TOE.}
        \item[ALC\_FLR.2.6C]{The procedures for processing reported security flaws shall ensure that
            any reported flaws are remediated and the remediation procedures issued to TOE users.}
        \item[ALC\_FLR.2.7C]{The procedures for processing reported security flaws shall provide
            safeguards that any corrections to these security flaws do not introduce any new flaws.}
        \item[ALC\_FLR.2.8C]{The flaw remediation guidance shall describe a means by which TOE users
            report to the developer any suspected security flaws in the TOE.}
    \end{description}
\end{quote}
\todo[inline]{Célja a flaw reporting procedures-nek}

\subsection{ALC\_FLR.3 - Systematic flaw remediation}

Az előző szinthez képes az alábbi szempontok adódtak hozzá a jelenlegi szinthez:
\begin{quote}
    \begin{description}
        \item[ALC\_FLR.3.6C]{The flaw remediation procedures shall include a procedure requiring
            timely response and the automatic distribution of security flaw reports and the
            associated corrections to registered users who might be affected by the security flaw.}
        \item[ALC\_FLR.3.10C]{The flaw remediation guidance shall describe a means by which TOE
            users may register with the developer, to be eligible to receive security flaw reports
            and corrections.}
        \item[ALC\_FLR.3.11C]{The flaw remediation guidance shall identify the specific points of
            contact for all reports and enquiries about security issues involving the TOE.}
    \end{description}
\end{quote}
\todo[inline]{Célja a systematic flaw remediationnak.}

\subsection{Flaw remediation megvalósítása}

\todo[inline]{Procedúra JIRA-ba bevezetve}

\section{Fejlesztési- és karbantartási folyamatok (ALC\_LCD)}
Ez a szekció megmutatja, hogy a jelenlegi, gyengén irányított folyamatainkat miért lenne előnyös
magasabb szinten felügyelni, és milyen kezdeti lépések lennének megfelelőek ennek elérése irányában.

A \emph{Common Criteria} szempontjából ez azért fontos kérdés, mivel ekkor a
\emph{funkcionális biztonsági követelmények (Security Functional Requirements, SFR)} megvalósulása
egyáltalán nem garantált, amely nyilvánvalóan elfogadhatatlan, mivel azért (is) vezettük be a
\emph{Common Criteria}-t, hogy a termékünk biztonságosabb legyen.

A folyamat gyakorlati alkalmazásának egy lényeges követelménye, hogy a fejlesztést ne akadályozza
számottevően, mivel ilyenkor fennáll a motiváció, hogy ezeket valamilyen módon megkerüljük. Ezzel
ekvivalens követelmény, ha valamilyen módon minimalizáljuk a fejlesztő adminisztrációval eltöltött
idejét, amelyből következik, hogy olyan adminisztrációs eszközt használunk, amely nagyfokú
automatizálást tesz lehetővé, és a nehezen automatizálható feladatok elvégzésére egy nagyon egyszerű
felületet biztosít.

\subsection{Jelenlegi folyamatok}
A jelenlegi munkafolyamatokat az alábbi gráfok jellemzik, ahol az egyes csomópontok a fentebb
említett állapotok közül lehetségesek, az élek pedig az engedélyezett átmeneteket jelentik.

\begin{figure}[h]
    \centering
    \includegraphics[width=\textwidth, height=0.4\textheight, keepaspectratio]{figures/oldfeature.png}
    \caption{A jelenlegi, megengedő fejlesztési folyamat}
    \label{fig:oldfeature}
\end{figure}
\FloatBarrier
\pagebreak[3]

\begin{figure}[h]
    \includegraphics[width=\textwidth, height=0.4\textheight, keepaspectratio]{figures/oldmt.png}
    \centering
    \caption{A jelenlegi, megengedő karbantartási folyamat}
    \label{fig:oldmt}
\end{figure}

\FloatBarrier
\pagebreak[3]
\subsection{Szigorítások a folyamatokon}

Mint ahogy az a gráfokból leolvasható, sajnos tetszőleges két állapot között megengedett átmenet,
ezért például lehetséges egy hiba javításánál a szakmai lektorálás kihagyása, amely jobb esetben
csak a kód minőségének romlásához, legrosszabb esetben viszont kártékony kód kódbázisba kerüléséhez
vezethet.

Első közelítésben érdemes korlátozni a lehetséges átmeneteket az alábbiakra:

\begin{figure}[h]
    \includegraphics[width=\textwidth, keepaspectratio]{figures/newfeature.png}
    \centering
    \caption{Korlátozott fejlesztési folyamat}
    \label{fig:newfeature}
\end{figure}

\begin{figure}[h]
    \includegraphics[width=\textwidth, keepaspectratio]{figures/newmt.png}
    \centering
    \caption{Korlátozott karbantartási folyamat}
    \label{fig:newmt}
\end{figure}

\FloatBarrier

Az átmenetek létrehozásával lehetőségünk nyílt \emph{JIRA}-ban az egyes átmenetekhez
\begin{itemize}
    \item \emph{feltétel}eket (olyan előfeltételek, amelyek teljesülése engedi az állapot
        végrehajtását),
    \item \emph{validator}okat (olyan ellenőrzések, amelyek az átmenet végrehajtása előtt
        értékelődnek ki),
    \item és \emph{post function}-öket (olyan műveletek, amelyek az átmenet végrehajtása után
        lefutnak)
\end{itemize}
definiálni. Ez az eszköztár segíthet abban, hogy ne csupán lassítsuk a kritikus állapotok
átugrásának kísérletét, de akár kényszeríthetjük, hogy más emberekkel együtt kelljen működnie ahhoz,
hogy a szándékolt változtatást végrehajtsa. Feltétlezihetjük, hogy a belső támadó számára ez már
elég költséges és/vagy kockázatos ahhoz, hogy inkább más módszert keressen.

Ez például megvalósítható úgy, hogy eltárolom azokat a fejlesztőket, akik létrehozták a megoldást
(azaz azokat, akik valaha ,,\emph{To do}''-ból ,,\emph{In progress}''-be mozgatták az adott jegyet),
és csak annak engedem a tesztelésen és a lektoráláson való továbbengedést, akik nincsenek benne az
előbb létrehozott listában.

További kényelmi funkciókat is megalapoztunk az átmenetek szétválasztásával, mint például azt, hogy
végrehajtásukkor van lehetőségünk a JIRA felhasználótól információt bekérni.  Ez önmagában nem tűnik
nagy előnynek, de ha belegondolunk, akkor például a karbantartás tanulási görbéje kevésbé lett
meredek, hiszen így már nem kell ismerni, globálisan a folyamatot, elegendő lokális döntéseket
meghozni (átment-e a módosítás a teszteken, vagy sem), illetve nem kell ismerni például azt sem,
hogy mielőtt a tesztelőknek odaadnám a módosításaimat, létre kell hoznom egy teljes
telepítőcsomagot, ha a \emph{JIRA} rákérdez az átmenet végrehajtása előtt, hogy milyen elérési
útvonalon találja meg a generált telepítőcsomagot. Ez a megközelítés gyakorlatilag
a \emph{fail-fast} \cite{FailFast} szoftverfejlesztési technika folyamatokra vetített változata.

\pagebreak[3]
\section{Configuration Management}
A \emph{Configuration Management} (továbbiakban: \emph{CM}) egy olyan rendszer, amely arra hivatott,
hogy azokat a változtatásokat nyomon tudja követni, amelyek hatással lehetnek a vizsgált termékre.
Amennyiben csak jól meghatározott, autorizált változtatásokat engedélyezünk, biztosíthatjuk a
termékünk integritását.

\subsection{Configuration Management Scope}
\todo[inline]{Egyes pontok kifejtése}
Természetesen mielőtt egyáltalán bevezetnénk a \emph{CM}-et, szükséges meghatározni annak hatókörét.
Ez a szekció meghatározza azokat a dolgokat, amelyeket feltétlenül szükséges a \emph{CM} rendszerben
kezelnünk, a szinteket is ez határozza meg (azaz magasabb szinten többet kell lefednie
a \emph{CM}-nek).

\pagebreak[3]
\subsubsection{ALC\_CMS.1 - TOE CM coverage}
\begin{quote}
    \begin{description}
        \item[ALC\_CMS.1.1D]{The developer shall provide a configuration list for the TOE.}
        \item[ALC\_CMS.1.1C]{The configuration list shall include the following: the TOE itself; and
            the evaluation evidence required by the SARs.}
        \item[ALC\_CMS.1.2C]{The configuration list shall uniquely identify the configuration
            items.}
    \end{description}
\end{quote}



\pagebreak[3]
\subsubsection{ALC\_CMS.2 - Parts of the TOE CM coverage}
\begin{quote}
    \begin{description}
        \item[ALC\_CMS.2.1C]{The configuration list shall include the following: [\ldots] \emph{and
            the parts that comprise the TOE}.}
        \item[ALC\_CMS.2.3C]{For each TSF relevant configuration item, the configuration list shall
            indicate the developer of the item.}
    \end{description}
\end{quote}

\pagebreak[3]
\subsubsection{ALC\_CMS.3 - Implementation representation CM coverage}
\begin{quote}
    \begin{description}
        \item[ALC\_CMS.3.1C]{The configuration list shall include the following: [\ldots];
            \emph{and the implementation representation}.}
    \end{description}
\end{quote}

Mint látható, csupán annyit változott a követelmény, hogy az implementációs reprezentációt (, azaz
gyakorlatilag a termék forráskódját) \emph{CM}-ben kezeljük. Mivel már minden forráskódot
a \emph{git} verziókezelővel követünk, ezért ha ezt a rendszert beleértjük a CM-be, akkor könnyedén
teljesíthetjük ezt a követelményt.

\pagebreak[3]
\subsubsection{ALC\_CMS.4 - Problem tracking CM coverage}
\begin{quote}
    \begin{description}
        \item[ALC\_CMS.4.1C]{The configuration list shall include the following: [\ldots]; \emph{and
            security flaw reports and resolution status}.}
    \end{description}
\end{quote}

A \emph{flaw remediation} részben arra a konklúzióra jutottunk, hogy a biztonsági hibákat
\emph{JIRA}-ban \cite{JIRA} követjük, ezért a \emph{CM} rendszerhez érdemes hozzávennünk magát
a \emph{JIRA}-t is, pontosan úgy, mint ahogy az előző pontban tettük. Mivel azonban már három
független rendszer összekapcsolásánál járunk, egyre jobban kell figyelnünk a megfelelő
szinkronizáció kialakításánál.

\pagebreak[3]
\subsubsection{ALC\_CMS.5 - Development tools CM coverage}
\begin{quote}
    \begin{description}
        \item[ALC\_CMS.5.1C]{The configuration list shall include the following: [\ldots]; security
            flaw reports and resolution status; \emph{and development tools and related
            information}.}
    \end{description}
\end{quote}

A fejlesztői eszközök (olyan eszközök, amelyek részt vesznek a termék létrehozásában) és azok
konfigurációi definíció szerint olyan dolgok, amelyek módosításait \emph{CM} rendszerben erősen
ajánlott követnünk, hiszen ha például lehetőségünk van autorizálatlanul az egyes fordítóprogramokat
tetszőleges binárisra lecserélnünk, máris tetszőlegesen módosíthatjuk a \emph{TOE}-t annak
generálása során, vagy ha ugyan az eszközök változásait követjük, de a konfigurációjáét nem, akkor
például egyes fordítói ellenőrzések kikapcsolásával, olyan hibákat engedhetünk a kódban, amelyeket
már a fordítás során is észrevehettünk volna.

Ezt a követelményt továbbfejleszthetjük úgy, hogy a \emph{Infrastructure as Code, IaC} paradigmát
szem előtt tartva olyan formában írjuk le, amely alkalmas a fejlesztési környezet könnyű
reprodukciójához. Ilyen például a \emph{syslog-ng} fordítására alkalmas \emph{Docker} konténert
leíró \emph{Dockerfile}. \cite{syslogngenv}

Ennek a követelménynek a betartatásánál könnyedén túlzásokba is eshetünk úgy, hogy azt
a környezetet, amelyben a implementációs reprezentáción dolgozunk, fejlesztőinek tekintjük, és
korlátozzuk a fejlesztőket ennek használatára. Belátható, hogy ez feleslegesen mehet
a termelékenység rovására.

\pagebreak[3]
\subsection{Configuration Management Capabilities (ALC\_CMC)}

A \emph{CM} másik lényeges összetevője annak definiálása, hogy maga a rendszer milyen
változtatásokat engedélyez a felügyelt elemein. Erre szolgál az \emph{ALC\_CMC} alosztály.
Az előző alosztályokhoz hasonlóan a szintek meghatározása az ellenőrzések szigorúságával van
összefüggésben.

\subsubsection{ALC\_CMC.1 - Labelling of the TOE}
\begin{quote}
    \begin{description}
        \item[ALC\_CMC.1.1D]{The developer shall provide the TOE and a reference for the TOE.}
        \item[ALC\_CMC.1.1C]{The TOE shall be labelled with its unique reference.}
    \end{description}
\end{quote}

Az \emph{EAL1} szint elérése ennél az alosztálynál egy egészen egyszerű feladatnak tűnik, hiszen
csupán annyi a feladat, hogy a termék egy adott állapotát egy egyedi cimkével ellássuk, és mivel
a verziószám készítése manapság elterjedt, ezért adódik, hogy legyen ez az egyedi cimkénk.

Sajnos ez csak addig ennyire egyszerű, amíg nem jutunk el a kiadás létrehozásáig: vannak olyan
teszteseteink (továbbiakban: \emph{Release teszt}), amelyek futtatása költséges, és/vagy különleges
eszközökön kell azt futtatni. Ezt korábban úgy oldottuk fel, hogy az ilyen tesztek futtatását
a kiadás létrehozásáig elodáztuk.  Amennyiben anomáliát észlelünk, a javítástól újraindítjuk
a termékfejlesztés folyamatát, és \emph{a cimkét átmozgatjuk a javított állapotra}, amellyel azonnal
megsérthetjük a cimkék egyediségének feltételét , ha az átmozgatás nem mindenütt történik meg,
továbbá ez a fejlesztők között is könnyedén félreérthetésekhez vezethet.

Adódik a javaslat is, hogy külön jelöljük azokat a verziókat, amelyek eljutottak a Release tesztig,
például a széles körben használt \emph{sorszámozott Release Candidate}, röviden \emph{rc}
szuffixszel.
Alternatív módszerek lehetnek az alábbiak:
\begin{itemize}
    \item \emph{git describe} használata: \\
        Például: \emph{syslog-ng-3.7.2-834-gdb3795a}, amely az alábbi formátumnak felel meg: \\
        \emph{<commit-tól a szülők felé lépdelve az első elérhető tag>-<az elért tagtől hány commit
            történt a paraméterben kapott commit-ig>-<verziókezelőt leíró karakter><paraméterben
        átadott commit commitid-ja>}
\end{itemize}

\pagebreak[3]
\subsubsection{ALC\_CMC.2 - Use of a CM system}

\begin{quote}
    \begin{description}
        \item[ALC\_CMC.2.2D]{The developer shall provide the CM documentation.}
        \item[ALC\_CMC.2.3D]{The developer shall use a CM system.}
        \item[ALC\_CMC.2.2C]{The CM documentation shall describe the method used to uniquely
            identify the configuration items.}
        \item[ALC\_CMC.2.3C]{The CM system shall uniquely identify all configuration items.}
    \end{description}
\end{quote}
Belátható, hogy a \emph{git} verziókezelő rendszer széleskörű használata kezdetben megfelelő
lehet CM rendszernek is.
\todo[inline]{Miért jó a git?}

\pagebreak[1]
\subsubsection{ALC\_CMC.3 - Authorisation controls}
\begin{quote}
    \begin{description}
        \item[ALC\_CMC.3.4C]{The CM system shall provide measures such that only authorised changes
            are made to the configuration items.}
        \item[ALC\_CMC.3.5C]{The CM documentation shall include a CM plan.}
        \item[ALC\_CMC.3.6C]{The CM plan shall describe how the CM system is used for the
            development of the TOE.}
        \item[ALC\_CMC.3.7C]{The evidence shall demonstrate that all configuration items are being
            maintained under the CM system.}
        \item[ALC\_CMC.3.8C]{The evidence shall demonstrate that the CM system is being operated in
            accordance with the CM plan.}
    \end{description}
\end{quote}

Ez az a pont, ahol a \emph{CM} használata értelmet nyer, mivel innentől már nem engedélyez
tetszőleges módosításokat a kódbázison, ezáltal csökkentve a lehetőségét a \emph{TOE} rosszindulatú
módosításának.

Mivel előző pontban beláttuk a \emph{git} \emph{CM}-ként való  használatának előnyét, és
a \emph{git}-et elsődlegesen forráskódok kezelésére használják, ezért visszavezethetjük
a problémánkat arra, hogy hogyan \emph{szokás} a forráskód módosításainál eldönteni, hogy az
autorizált-e.  Észrevehetjük, hogy a nyílt, közösség által fejleszett projektek ezt meg tudják úgy
oldani, hogy:
\begin{itemize}
    \item{csak olyan változtatások kerülhetnek be a kódbázisba, amelyet egy szűkebb csoport
        elfogad,}
    \item{mégsem szűkül a lehetséges kontribútorok halmaza, azaz gyakorlatilag bárki
        kontributálhat.}
\end{itemize}

Ezen feltételek egy lehetséges megoldása az úgynevezett \emph{Forking Workflow}-t használata
a fejlesztésre, amely röviden az alábbi lépésekből áll:
\begin{enumerate}
    \item{Létrejön egy ún. \emph{blessed repository}, amely a forráskód referenciaváltozatát
        tartalmazza.}
    \item{Az egyes fejlesztők leklónozzák ezt a \emph{repository}-t sajátként (ez a művelet úgy is
        ismert, mint \emph{fork}olás)}
    \item{A fejlesztő a saját \emph{repository}-jában elkészíti a módosításait.}
    \item{Valamilyen módon értesíti a \emph{blessed repository} tulajdonosát
        (nevezzük ezt a személyt \emph{integrátornak}) az elkészített változtatásokról.}
    \item{Az integrátor elfogadja, vagy elveti a javasolt változtatásokat.}
\end{enumerate}

\begin{figure}[h]
    \centering
    \includegraphics[width=\textwidth, height=0.25\textheight, keepaspectratio]{figures/forkingworkflow.png}
    \caption{Példa az egyes kontribútorok és a \emph{blessed repository} viszonyáról \cite{ForkingWorkflow}.}
\end{figure}
\FloatBarrier

Természetesen ahhoz, hogy ez működjön, meg kell tiltani a \emph{blessed repository} írását a többi
fejlesztő által(, vagy legalábbis csak kivételes esetekben szabad ezeket engedélyezni).

A \emph{syslog-ng}-nél a közösség által fejlesztett változatban már korábban igazodni kellett ehhez
a munkafolyamathoz, a kereskedelmi célú változatában ezt a gondolkodásmódot még erősíteni szükséges,
mivel általában a módosítások integrálásának kérelme (továbbiakban: \emph{pull/merge request}) meg
szokott történni, de ez a \emph{blessed repository} egyik ágának a másikára szokott történni, azaz
semmi sem garantálja, hogy egy fejlesztő ne csak az általa létrehozott \emph{branch}-et tudja írni.

\pagebreak[1]
\subsubsection{ALC\_CMC.4 - Production support, acceptance procedures and automation }
\begin{quote}
    \begin{description}
        \item[ALC\_CMC.4.4C]{The CM system shall provide \emph{automated} measures such that only
            authorised changes are made to the configuration items.}
        \item[ALC\_CMC.4.5C]{The CM system shall support the production of the TOE by automated
            means.}
        \item[ALC\_CMC.4.8C]{The CM plan shall describe the procedures used to accept modified or
            newly created configuration items as part of the TOE.}
    \end{description}
\end{quote}

Azaz ez a szekció a magasabb szintű automatizálás megvalósítására hivatott, mivel az
alapfeltételezés az, hogy az emberi tényező megbízhatósága lényegesen alacsonyabb lehet egy
megegyező munkát végző géppel, főként az amúgy könnyedén automatizálható feladatoknál.  Mint ahogy
az a követelményekből látszik, a \emph{Common Criteria} csak az autorizált változtatások bekerülését
szabályozná automatikusan, viszont érdemes továbbgondolni a feladatot, mivel ha az \emph{4.8C}-ben
említett pontot kellőképpen formálisan írjuk le, akkor abból könnyedén programkód készülhet, és
végsősoron az ellenőrzések egy részhalmazát automatikusan megtehetjük.
Az alosztály továbbá megköveteli a TOE automatikus létrehozhatóságát is. Ez összhangban áll a
\emph{Continuous Integration} használatával. \todo[inline]{moar}

A \emph{syslog-ng} közösség által fejlesztett változatánál a legnagyobb elmaradás az elfogadás
szükséges feltételeinek dokumentálása, és az automatizálható ellenőrzések automatizálása.

Van az ellenőrzéseknek egy olyan fajtája, amelyeket csak az újonnan beérkező kódra szeretnénk
alkalmazni, mint például a kódstílus egységesítése automatikus formázók segítségével:
mind annak azonnali alkalmazása, mind a stílusleíró változásával járó újraalkalmazás túl nagy
különséget eredeményezne a formázás előtti és utáni kód között (ha például a jelenlegi \emph{master}
ágon szeretném az összes \emph{C} nyelven írt forrásfájlt formázni, akkor az önmagában egy $16375$
sor hozzáadásával és $15982$ sor törlésével járna). Ennek teljes körű szakmai lektorálása
nehézkes az átnézendő kód mérete miatt, a lektorálatlan kódról pedig feltételeznünk kell, hogy
kártékony módosításokat tartalmazhat, hiszen az esetek nagy részében a \emph{syslog-ng} \emph{root}
felhasználóként fut, akinek a jogköre már egészen széles szokott lenni ahhoz, hogy érdemes legyen
megszemélyesíteni.

Az ilyen transzformációk miatt felmerült egy olyan könnyűsúlyú (ezáltal CI rendszerbe integrálható)
eszköz létrehozása, amely képes ellenőrzéseket és átalakításokat végrehajtani csak és kizárólag
néhány kódmódosítási egységen (pontosabban az adott verziókezelő által atominak tekintett
módosításokon, másnéven \emph{commit}-okon), míg a kódbázis maradékát érintetlenül hagyja.
Ellenőrzések esetén így megtalálhatjuk az első olyan \emph{commit}-ot, amely egy adott szempontból
nem megfelelő, transzformációk esetén pedig \emph{commit}-ról \emph{commit}-ra tehetünk javaslatokat
a javításra.

\todo[inline]{Tool beillesztése?}

Fontos megjegyezni, hogy az eszköz hatékony használatához elengedhetetlen a \emph{commit}-ok
jólszervezettsége, hiszen minél kisebb egy módosítás, annál könnyebb a problémáit észrevenni,
illetve ellenőrizni, hogy az eszközeink úgy módosították a kódot, ahogy azt elvártuk.

\subsubsection{ALC\_CMC.5 - Advanced support}
\begin{quote}
    \begin{description}
        \item[ALC\_CMC.5.3C]{The CM documentation shall justify that the acceptance procedures
            provide for an adequate and appropriate review of changes to all configuration items.}
        \item[ALC\_CMC.5.7C]{The CM system shall ensure that the person responsible for accepting a
            configuration item into CM is not the person who developed it.}
        \item[ALC\_CMC.5.8C]{The CM system shall identify the configuration items that comprise the
            TSF.}
        \item[ALC\_CMC.5.9C]{The CM system shall support the audit of all changes to the TOE by
            automated means, including the originator, date, and time in the audit trail.}
        \item[ALC\_CMC.5.10C]{The CM system shall provide an automated means to identify all other
            configuration items that are affected by the change of a given configuration item.}
        \item[ALC\_CMC.5.11C]{The CM system shall be able to identify the version of the
            implementation representation from which the TOE is generated.}
    \end{description}
\end{quote}

Ez a

\pagebreak[2]
\section{Termék leszállítása (ALC\_DEL)}

Ennek a szekciónak a célja olyan intézkedések/követelmények megfogalmazása, amelyek biztosítják,
hogy a felhasználóhoz biztonságosan jut el vizsgált termék. Természetesen ez az osztály nem érinti a
közösség által fejlesztett változatot, hiszen ott rendelkezésre áll mind a forráskód, mind azok az
utasítások, amelyeket végrehajtva a forráskódból futtatható szoftvert állít elő, ezek védelmével
pedig a korábban említett \emph{CM} rendszer már foglalkozott.

Habár a Common Criteria eléggé szűkszavúan fogalmazza meg a követelményeket (csupán annyit ír elő,
hogy dokumentáljuk, és a dokumentáció alapján szállítsuk a TOE-t), a segítségül szolgáló
\emph{Application notes} alatt találhatunk gondolatébresztő szempontokat:

\begin{quote}
    The delivery procedures should consider, if applicable, issues such as:
    \begin{itemize}
        \item ensuring that the TOE received by the consumer corresponds precisely to the evaluated
            version of the TOE;
        \item avoiding or detecting any tampering with the actual version of the TOE;
        \item preventing submission of a false version of the TOE;
        \item avoiding unwanted knowledge of distribution of the TOE to the consumer: there might be
            cases where potential attackers should not know when and how it is delivered;
        \item avoiding or detecting the TOE being intercepted during delivery; and
        \item avoiding the TOE being delayed or stopped during distribution.
    \end{itemize}
\end{quote}



\todo[inline]{Producttól elkérni a folyamatot}

\section{Development Security}

\todo[inline]{SecMan}
