\chapter{A metodologia alkalmazása}

\section{Termékfejlesztési folyamatok vizsgálata}
\subsection{Észlelt hibák biztonsági vizsgálata}

Lsd: Flaw remediation (ALC\_FLR)

\subsection{Fejlesztési- és karbantartási folyamatok}

\subsection{Configuration Management}
A \emph{Configuration Management} (továbbiakban: CM) egy olyan rendszer, amely arra hivatott,
hogy azokat a változtatásokat nyomon tudja követni, amelyek hatással lehetnek a vizsgált termékre.
Amennyiben csak jól meghatározott, autorizált változtatásokat engedélyezünk, biztosíthatjuk a
termékünk integritását.
Természetesen mielőtt egyáltalán bevezetnénk a \emph{CM}-et, feltétlenül szükséges meghatározni
annak hatókörét. Erre szolgál a \emph{ALC\_CMS} szekció.

\subsubsection{Configuration Management Scope}
\begin{itemize}
    \item{ALC\_CMS.1}
    \item{ALC\_CMS.2}
    \item{ALC\_CMS.3}
    \item{ALC\_CMS.4}
    \item{ALC\_CMS.5}
\end{itemize}

\subsubsection{Configuration Management - termék verziózása}
Az EAL1 szint elérése ennél az alosztálynál egy egészen egyszerű feladatnak tűnik, hiszen csupán
annyi a feladat, hogy a termék egy adott állapotát egy egyedi cimkével ellássuk, és mivel a
verziószám készítése manapság elterjedt, ezért adódik, hogy legyen ez az egyedi cimkénk.

Sajnos ez csak addig ennyire egyszerű, amíg nem jutunk el a kiadás létrehozásáig:
vannak olyan teszteseteink (továbbiakban: \emph{Release teszt}),
amelyek futtatása költséges, és/vagy különleges helyeken kell azt
futtatni. Ezt a problémát úgy oldottuk fel, a létrehozott kiadáson hajtjuk végre ezeket.
Amennyiben anomáliát észlelünk, a javítástól újraindítjuk a termékfejlesztés folyamatát,
és \emph{a cimkét átmozgatjuk a javított állapotra}, amellyel azonnal megsérthetjük az egyedi
cimkét, ha az átmozgatás nem mindenütt történik meg.

Adódik a javaslat is, hogy külön jelöljük azokat a verziókat, amelyek eljutottak a Release tesztig,
például a széles körben használt \emph{sorszámozott Release Candidate}, röviden \emph{rc} szuffixszel.

\subsubsection{Configuration Management - CM rendszer használata}
Belátható, hogy a \emph{git} verziókezelő rendszer széleskörű használata kezdetben megfelelő
lehet CM rendszernek is.

\subsubsection{Configuration Management - Autorizáció}
\subsubsection{Mi változott az előző szinthez képest?}
\begin{qoute}
\end{quote}
Ez az a pont, ahol a CM rendszer használata értelmet nyerhet, hiszen pont azzal a szándékkal
vezetjük be, hogy csak az autorizált változtatásokat engedélyezze.

Az előző pontban beláttuk a \emph{git} használatának előnyét, és a \emph{git}-et elsődlegesen
forráskódok kezelésére használják, ezért visszavezethetjük a problémánkat oda, hogy hogyan
\emph{szokás} a forráskód módosításainál eldönteni, hogy az autorizált-e.
Észrevehetjük, hogy a nyílt, közösség által fejleszett projektek ezt meg tudják
úgy oldani, hogy:
\begin{itemize}
    \item{csak olyan változtatások kerülhetnek be a kódbázisba, amelyet egy szűkebb csoport elfogad,}
    \item{továbbá megmarad a lehetőség arra, hogy akárki képes legyen kontributálni.}
\end{itemize}
Azért képesek erre, az úgynevezett \emph{Forking Workflow}-t használják a fejlesztésre, amely röviden
az alábbi lépésekből áll:
\begin{enumerate}
    \item{Létrejön egy ún. \emph{blessed repository}, amely a forráskód referenciaváltozatát
        tartalmazza.}
    \item{Az egyes fejlesztők leklónozzák ezt a \emph{repository}-t sajátként.}
    \item{A fejlesztő a saját \emph{repository}-jában elkészíti a módosításait.}
    \item{Valamilyen módon értesíti a \emph{blessed repository} tulajdonosát
        (nevezzük ezt a személyt \emph{integrátornak}) az elkészített változtatásokról.}
    \item{Az integrátor elfogadja, vagy elveti a javasolt változtatásokat.}
\end{enumerate}
Természetesen ahhoz, hogy ez működjön, meg kell tiltani a \emph{blessed repository} írását a
többi fejlesztő által, vagy legalábbis csak kivételes esetekben szabad ezeket engedélyezni.

A \emph{syslog-ng}-nél a közösség által fejlesztett változatban már korábban igazodni kellett
ehhez a munkafolyamathoz, a kereskedelmi célú változatában ezt a gondolkodásmódot még erősíteni
szükséges, mivel általában a módosítások integrálásának kérelme (továbbiakban: \emph{pull/merge request})
meg szokott történni, de ez a \emph{blessed repository} egyik ágának a másikára szokott történni,
azaz semmi sem garantálja, hogy egy rosszindulatú fejlesztő képtelen legyen tetszőleges módosítást
végrehajtani a kódbázison.

\subsubsection{Configuration Management - Production support, acceptance procedures, automation}
\subsubsection{Mi változott az előző szinthez képest?}
\begin{quote}
    \begin{description}
        \item[ALC\_CMC.4.4C]{The CM system shall provide \emph{automated} measures such that only authorised
            changes are made to the configuration items.}
        \item[ALC\_CMC.4.5C]{\emph{The CM system shall support the production of the TOE by automated means.}}
        \item[ALC\_CMC.4.8C]{\emph{The CM plan shall describe the procedures used to accept modified or newly created
            configuration items as part of the TOE.}}
    \end{description}
\end{quote}
Azaz ez a szekció a magasabb szintű automatizálás megvalósítására hivatott

\subsection{Delivery}

Ennek a szekciónak a célja olyan intézkedések/követelmények megfogalmazása, amelyek biztosítják, hogy a felhasználóhoz biztonságosan jut el a termékünk.

\section{Development Security}

