\chapter{Háttérismeretek}

\section{A \emph{syslog-ng}-ről}

A \emph{syslog-ng} egy 1998 óta fejlesztett naplózó \emph{daemon}. Eredetileg nyílt forráskódú
szoftverként indult, de időközben létrejött egy kereskedelmi változata is, és habár a két kódbázis
között még erős a kapcsolat, de gyakorlatilag a különbségük annyira jelentős, hogy inkább
tekinthetjük őket két, különálló terméknek, melyek újraegyesítése jelenleg is folyamatban van.

A szoftver gerince \emph{C} nyelven íródott, a modulok fejlesztése pedig történhet más, magasabb
szintű nyelveken is. Az alacsony szintű nyelven megírt gerinc annak is a következménye, hogy
a szoftver teljesítménye mindig is elsődleges szempont volt.

A technológiai adósságok\footnote{Egy adott problémára adható megoldások az ,,egyszerű, de nem
    igényes'' -- a ,,letisztult, igényes, de munkaigényes'' skálán helyezkednek el.
    \emph{Technológiai adósság}ot akkor veszünk fel, amikor a alacsonyabb minőségű megoldást
    választunk annak gyors létrejöttéért, és minden egyes alkalommal, amikor ezzel a kóddal újra
foglalkozni kell (további fejlesztéseket, karbantartást kell eszközölni rajta), többletmunkával
fizetünk. \cite{magnusson2014technology}} megfelelő kezelése még az újonnan felépített szoftvernél
sem mondható triviális feladatnak, de esetünkben a
\begin{itemize}
    \item hosszú múlttal rendelkező (tehát hosszú idő alatt gyűlhet/kamatozhat az adósság),
    \item teljesítménycentrikus (azaz nemtriviális (mikro)optimalizációkat engedélyező kód), és
    \item alacsony szintű nyelven írt (ahol a helyes működés eléréséhez magas szakértelem szükséges)
\end{itemize}
szoftver miatt ez nehéz feladatnak tűnik. Másik jellemzője a technológiai adósságnak, hogy már csak
akkor válik érezhetővé a jelenléte, amikor az már jelentős mértékben összegyűlt.  Ilyenkor érdemes
megállni egy pillanatra, és az adósságot törleszteni. Ennek érkezett el az ideje.


\subsection{A fejlesztéshez használt segédeszközök}

Ebben az alszekcióban szeretném bemutatni azokat a fontosabb eszközöket, amelyek a fejlesztéshez nap
mint nap használunk.

\begin{description}
    \item[JIRA] {Az Atlassian által fejlesztett, elsősorban hibajegyek kezelésére, és az agilis
        munkafolyamatok támogatására használatos szoftver. Esetünkben a \emph{VersionOne}-t váltotta
        le.}
    \item[Zorp Build System (ZBS)] {A ZBS egy házon belül fejlesztett eszköz, célja a forráskódból
        a telepíthető binárisok, csomagok előállítása, azaz elfedni a különböző platformok (CPU
        architektúrák, futtató operációs rendszer, stb.) közötti fordítási és csomagolási
        folyamatokban lévő különbségeket. Mivel vannak olyan támogatott platformok, amelyek
        kevésbé szokványos hardvereket igényelnek, és egy ilyen gépen nem történhet párhuzamosan
        több fordítás, ezért a ZBS-nek ezeket az erőforrásokat is kezelnie kell. }
    \item[Zorp Test Suite (ZTS)] { A ZTS egy belső teszteléshez használt keretrendszer, amely
        ZBS-hez hasonlóan képes több platformon különféle teszteket végrehajtani. Mivel az egyes
        tesztesetek különleges infrastruktúrákat igényelhetnek, ezért a ZTS feladata az
        infrastruktúra felépítése, és az infrastruktúra egyes elemeinek lefoglalása, tesztesetek
        közötti újrahasznosítása.}
    \item[git] {A git már egy nagy múltra visszatekintő (először 2005-ben adták ki), eredetileg a
        Linux kernel fejlesztését segítő, elosztott verziókezelő rendszer. Elosztottságának
        köszönhetően a nyílt forráskódú szoftverek világában az egyik legnagyobb mértékben
        adoptált, köztük a \emph{syslog-ng} fejlesztésénél is ezt használjuk mindkét változatában.
        (Kereskedelmi termékek fejlesztésénél ez annyira nem jellemző, nekünk viszont erősen
        ajánlott azért, hogy a két változat közötti kódmozgatás gördülékenyebb legyen.)}
    \item[GitHub] {Habár egy egyszerű tárhelynek tűnhet csupán, mégis az általa adott
        többletszolgáltatások a közösség által fejlesztett syslog-ng központi elemévé tette.
        Ilyen többletszolgáltatás a \emph{forkok} könnyű kezelése, a beépített hibajegykezelő,
        a \emph{pull request}-ek könnyű kezelése (azaz a GitHub-on lévő forkomból pár
        kattintással tudok kontribúciót küldeni az \emph{upstream repository}-ba), és a könnyű
        bővíthetősége, például lektorálást segítő eszközökkel, \emph{Continous Integration}-t
        megvalósító eszközökkel, statikus kódanalízist biztosító eszközökkel.
        Természetesen a névválasztás nem véletlen, nagyban épít a korábban említett \emph{git}
        verziókezelőre.}
\end{description}

\subsection{A fejlesztés során felmerülő problémák}

A teljesség igénye nélkül listázzuk a fejlesztést nehezítő, olykor értékes fejlesztői időt pazarló
problémákat:

\subsubsection{Lassan haladó fejlesztés}
Ez a probléma elsődlegesen a közösség által fejlesztett \emph{syslog-ng}-t érinti, azon belül is
arra a kérdésre keressük a választ, hogy miért tart sokáig egy beérkezett kontribúció integrálása
a kódbázisba. Ennek egy lehetséges magyarázata az, hogy nincsenek olyan automatikus ellenőrzések,
amelyek a kontribútor számára azonnali visszajelzést ad a kontribúció minőségéről, a lektorok
számára pedig ez felesleges kézi ellenőrzést jelent.

Egyik lényeges ilyen automatizmus a kód stílusának ellenőrzése, amely biztosítja, hogy kód
egységesen néz ki, és ha egy fejlesztő ezt már megszokta, akkor valamivel már otthonosabban mozog az
ismeretlen kódban is.  Érdekesség, hogy a \emph{CVE-2014-1266}\cite{cve20141266}, közismertebb nevén
\emph{goto fail;} sérülékenységet egy egyszerű \emph{code beautifier} figyelemfelkeltőbbé tette
volna azáltal, hogy a tabulálásokat (pontosabban fogalmazva \emph{indent}álásokat) kijavította
volna.

\subsubsection{Belső eszközök problémája}
Az egyik legnagyobb probléma (akár általánosságban is) a belső eszközökkel, hogy ezek mellékszálon
készülnek el, ezért általában alacsonyabb szinten dokumentáltak, amiből következik, hogy meredek
tanulási görbével rendelkeznek, amelyből pedig következik, hogy még kevesebben tudják csak
fejleszteni, mert a fejlesztők a használók közül kerülnének kiválasztásra.
Ezért ennek a területnek definíció szerint alacsony a \emph{busz faktor}a\footnote{\emph{busz
faktor:} egy (fejlesztő)csapaton belüli információ \emph{de}koncentráltságát jellemző mérőszám,
amely minél magasabb, annál jobb. Másképpen definiálva: legfejlebb hány ember veszíthető el úgy
a csapatból, hogy az még ne omoljon össze az egyedek elvesztéséből fakadó információveszteség miatt.
A definíció és az elnevezés közötti kapcsolat felfedezését az Olvasóra bízom.}, amely hosszabb távon
nézve kockázatos mind biztonsági (szélsőséges esetben egy ember ismer egy rendszert igazán, így az
ő módosításainak lektorálása nem hatékony), mind termelékenységi szempontból (hiszen ilyenkor ez az
ember szűk keresztmetszetként viselkedik).

Természetesen van ennél lényegesen rosszabban felépíthető belső eszköz is: nem találkoztam még olyan
fejlesztővel, aki ne írt volna egy kis szkriptet saját munkájának gyorsítására. A probléma ott
kezdődik, ha egy ilyen házilag barkácsolt szkriptet elkezdenek többen használni és ennek fejlesztése
annak toldozgatásában ki is merül. Olykor előfordulhat olyan szerencsétlen együttállás is, hogy 
az adott eszköz hibásan működik, de annak sem az elsődleges fejlesztője, sem annak tudása már nem
elérhető. Egy ilyen együttállás jelentősen lassíthatja a munkafolyamatokat.

\subsubsection{Törékeny fejlesztési infrastruktúra}
Ez leginkább abban mutatkozik meg, hogy vannak olyan nélkülönözhetetlen fejlesztői gépek, amelyeknél
azok létrehozási lépései elvesztek, illetve az idők során az azon végzett módosítások nem lettek
megfelelően dokumentálva. Amennyiben ez a fejlesztés szempontjából egy kritikus gép, belátható, hogy
az egy időzített bombaként viselkedik a rendszerben. Ennek a problémának egy tipikus áthidalása, ha
az adott gépről létezik visszaállítható biztonsági mentés, hiszen ekkor egy jól működő állapotra
visszaállhatunk, de az esetleges korszerűsítés (például frissebb operációs rendszerre való átállás)
még problémákat jelenthet.

Úgy gondoljuk, hogy ezek kijavítása ugyan kezdetben sok erőforrást igényelhetnek, de a későbbiekben
olyan mértékben egyszerűsítik a fejlesztők életét, hogy ez a befektetés megtérül.

\section{További termékbiztonságot elősegítő módszerek vizsgálata}

Időközben magasabb vezetési szintről felmerült az igény, hogy legyünk tisztában a termékeink
biztonsági állapotával, ezért erősen ajánlott egy olyan módszer kiválasztása, amely alkalmazható
kész \emph{appliance} termékre, és szoftverre egyaránt.

Habár a használt értékelő keretrendszer kiválasztása még a szakdolgozat létrejötte előtt megtörtént,
mégis érdemesnek tartom a hasonló, termékbiztonságot elősegítő módszerek bemutatását. Az egyes
megközelítések feltérképezéséhez és megvizsgálásához remek kiindulási alapot egy, a \emph{Common
Criteria}-t, \emph{Attack trees}-t, és \emph{Misuse cases}-t összehasonlító publikáció.
\cite{ThreeApproaches}

\subsection{Attack trees}

Az \emph{Attack trees}\cite{schneier1999attack} egy olyan módszer, amellyel egy adott támadást
elemezhetünk egyszerűen, egy faként reprezentálva az alábbi szabályok mentén:
\begin{enumerate}
    \item A fa gyökerében az az esemény áll, amely bekövetkezésekor a támadást sikeresnek tekintjük.
    \item A szülő és a gyerek eseménye között az a kapcsolat, hogy a szülő eseménye akkor, és csak
        akkor következik be, ha a gyerek eseménye is.  Összekapcsolt gyerekek esetén akkor és csak
        akkor, ha mindegyik gyerek eseménye bekövetkezett.
\end{enumerate}
Miután megtaláltuk az eseményeink függőségeit (pontosabban a további felbontásnak már nem lenne
értelme), lehetőségünk van ezeket különféle szempontokból elemezni, például az alábbi szempontok
mentén:
\begin{itemize}
    \item Lehetséges-e egyáltalán az esemény bekövetkezése?
    \item Mekkora költséggel jár előidézni az eseményt?
    \item Mekkora költséggel jár megelőzni az eseményt?
    \item Szükséges-e különleges felszerelés az előidézéshez?
\end{itemize}

\begin{figure}[h]
    \includegraphics[height=0.4\textheight]{figures/attacktrees.png}
    \centering
    \caption{Példa a felcímkézett Attack Trees használatára. \cite{schneier1999attack}}
\end{figure}

Az alábbi szempontok segíthetnek eldönteni, hogy megéri-e foglalkozni a védelem kialakításával,
szem előtt tartva, hogy mit feltételezünk a támadóról.

Habár a módszer kellően könnyedén értelmezhető (hiszen csupán egy fát használ) és könnyedén
alkalmazható, számunkra kevésbé használható, hiszen
\begin{itemize}
    \item{nem kellően széleskörű,}
    \item{megoldásokat egyáltalán nem szolgáltat,}
    \item{a teljesség nem garantált.}
\end{itemize}

\FloatBarrier
\subsection{Abuse/Misuse cases}

Az \emph{Misuse cases}\cite{alexander2003misuse} az ismert \emph{use case} diagramnak a kifordított
változata, azaz ahelyett, hogy azt modelleznénk, hogy egy adott szereplő milyen tevékenységeket
végezhet el, itt inkább azt modellezzük, hogy egy rosszindulatú szereplő milyen tevékenységeket
végezne el, azaz meghatározhatjuk azokat a tevékenységeket, amelyeket a \emph{rendszernek tiltania
kellene}.

\begin{figure}[h]
    \centering
    \includegraphics[width=\textwidth, height=0.5\textheight, keepaspectratio]{figures/misusecase.png}
    \caption{Példa a misuse cases diagram használatára.}
\end{figure}

\FloatBarrier

\section{Common Criteria}

A Common Criteria eredetileg egy olyan kezdeményezés volt, amely a már meglévő biztonsági
értékeléseket (köztük az európai \emph{Information Technology Security Evaluation Criteria (ITSEC)},
a \emph{Trusted Computer System Evaluation Criteria (TCSEC)}-t, és a
\emph{Canadian Trusted Computer Product Evaluation Criteria (CTCPEC)}-t) kívánta egyesíteni, ezáltal
elkerülve a korábbi ismételt kiértékeléseket egy nemzetközi piacra szánt termék esetén.  Az első
végleges változatát 1994-ben adták ki, 1999-ben a \emph{2.1}-es verziója az \emph{ISO/IEC 15408}-as
szabvánnyá vált. Legfrissebb változata a \emph{3.1 rev. 4}-es.

\subsection{Terminológia}

\begin{description}
    \item[Target of Evaluation (TOE)] {Az az entitás, amelyre a \emph{Common Criteria} által
        megfogalmazott követelmények teljesülését vizsgáljuk. Esetünkben ez a \emph{syslog-ng}-re,
        mint szoftvercsomagra vonatkozik.}
    \item[Protection Profile (PP)] { Az implementációfüggetlen leírása egy adott TOE-típus
        biztonsági követelményeinek, tulajdonságainak. }
    \item[Security Target (ST)] { Az implementációfüggő leírása az adott TOE biztonsági
        tulajdonságainak. Tartalmazza többek között a lehetséges fenyegetettségeket, feltétlezéseket
        a TOE környezetével kapcsolatban. Egy \emph{Security Target} megvalósíthat több
        \emph{Protection Profile}-t, azaz a \emph{PP} kvázi az \emph{ST} sablonjaként szolgál. }
    \item[Secutiry Functional Requirement (SFR)] { Az ST-ben található biztonsági célok lefordítása
        egy szabványosított nyelvezetre. Ez az első lépés ahhoz, hogy a magas absztakciós szinten
        lévő célból konkrét implementáció legyen. }
    \item[Security Assurance Requirement (SAR)] { Annak egy szabványosított leírása, hogy milyen
        intézkedéseket tettünk a fejlesztés során annak érdekében, hogy a \emph{TOE} teljesítse az
        \emph{SFR}-eket.}
    \item[Evaluation Assurance Level (EAL)]{Evaluation Assurance Level - a követelmények
        szigorúságának számszerűsített értéke, az egyes szintek jelentését lásd alább.}
\end{description}

\subsection{Az egyes EAL szintek jelentései}

Az \emph{Evaluation Assurance Level} (továbbiakban: EAL) egy egytől hétig tartó skálán egyre nagyobb
nyújt, pontosabban a magasabb szinteknél szigorúbb elvárásoknak kell megfelelni. A szigorúbb
elvárásoknak való megfelelőség óhatatlanul magasabb költségekkel is járhat, így az elérni kívánt
szint meghatározásánál figyelembe kell venni, hogy milyen előnyökkel és költségekkel jár annak
elérése.

\subsubsection{EAL1 - funkcionálisan tesztelt}
Ennek a szintnek az a célja, hogy a biztosítsuk azt, hogy a vizsgált termék funkcionalitása
a biztosított dokumentációval összhangban legyen. Éppen ezért a kiértékelés során jellemzően annyi
információt használunk fel, mint a termék egy felhasználója, ezáltal akár ő is megbizonyosodhat
a funkcionális és interfészspecifikáció betartásáról, a fejlesztőktől teljesen függetlenül.

Az esetleges biztonsági kockázatokat még nem tekintjük itt komolynak, és a biztonsági funkcionális
követelményeket csupán deklaráljuk ahelyett, hogy azok feltételezések és fenyegetettségekből
kerülnének levezetésre.

\subsubsection{EAL2 - strukturálisan tesztelt}
Az előző szinthez képest már magasabb szintű biztosítást nyújtunk a szint megvalósításával azáltal,
hogy a dokumentációkban nyújtunk tervezési információt, és megosztjuk az egyes tesztek eredményeit.
A biztosított dokumentációk területén annyi változás történt, hogy a specifikációnak itt már
részletesebbnek kell lennie, tartalmaznia kell egy áttekintést a vizsgált termék architektúrájáról,
teljes \emph{Security Target}-et, továbbá demonstrációt, hogy a vizsgált termék képes ellenállni
korlátozott kapacitású támadó ellen.

\subsubsection{EAL3 - módszeresen tesztelt és ellenőrzött}
Az EAL3 célja, hogy úgy biztosítsa a lehető legmagasabb bizonyosságot, hogy a már meglévő
fejlesztési szokásokat, módszereket ne változtassa meg nagy mértékben.
Annak érdekében, hogy megérthessük a TOE biztonsági működését, a biztosított dokumentációnak
tartalmaznia kell az előző szintnél részletesebb architekturális leírást TOE-ről.
További előrelépést jelent az \emph{EAL2}-höz képest, hogy
\begin{itemize}
    \item a biztonsági funkcionalitást átfogóbban kell tesztelni, azaz ott a tesztek lefedettségnek
        nagyobbnak kell lennie, és
    \item olyan folyamatokat és szabályozókat kell bevezetni, amelyekkel minimalizálni tudjuk a
        termékfejlesztés közbeni szabotázs lehetőségét.
\end{itemize}

\subsubsection{EAL4 - módszeresen tervezett, tesztelt és lektorált}
Az \emph{EAL4} az a legmagasabb elérendő szint, amely egy már meglévő termék esetén is még
gazdaságos lehet, mivel a bevezetett módosítások jelentősen szigorítják a folyamatokat, de speciális
tudást még nem igényel.

Az \emph{EAL3}-hoz képest úgy növeli a bizonyosságot, hogy:
\begin{itemize}
    \item a TOE biztonsági funkcionalitásához kapcsolódó implementáció reprezentációját
        (azaz a forráskódját) biztosítani kell,
    \item az előző szinten bevezetett szabotázst megakadályozó mechanizmusokat/folyamatokat
        fejleszteni kell, főként erősebb automatizálással,
    \item a TOE felépítésének modulárisnak kell lennie, és erről egy átfogó képet kell adni, és
    \item az interfész specifikáltságának teljesnek kell lennie.
\end{itemize}

\subsubsection{EAL5 - félformálisan tervezett és tesztelt}
\begin{description}
    \item[félformális]{Olyan korlátozott szintaxisú nyelvezet, amelynél a szemantika előre
        definiált, például ilyen lehet az UML osztálydiagramja.}
\end{description}

Az EAL5 az EAL4hez képest az alábbi követelmények betartatásával növeli
a bizonyosságot:
\begin{itemize}
    \item biztosítani kell a moduláris biztonsági funkciók terveit,
    \item olyan független sérülékenységi analízisre van szükség, amely képes demonstrálni a közepes
        kapacitású támadó elleni ellenállást,
    \item széleskörű automatizációt és folyamatirányítást kell eszközölni.
\end{itemize}

\subsubsection{EAL6 - formálisan verifikált}
Az EAL6 célja olyan TOE létrehozása, amely képes nagyértékű információk védelmére olyan környezetben
ahol a támadás kockázata szignifikáns.  Az \emph{EAL6} úgy növeli a bizonyosságot az \emph{EAL5}-höz
képest, hogy megköveteli félformálisan leírt funkcionális specifikációt és TOE dizájnt, a biztonsági
funkcióknak nem csak modulárisnak, hanem \emph{egyszerűnek és réteges felépítésű}nek kell lennie, és
a TOE biztonsági policy-jának formális modellre kell épülnie. A további minőségbiztosítás érdekében
strukturált fejlesztési folyamatra, és teljes automatizálásra van szükség.

\subsubsection{EAL7 - formálisan verifikált dizájn}
\begin{description}
    \item[formális]{Olyan félformális nyelvezet, melynek szemantikája megalapozott matematikai
        koncepciókon alapul.}
\end{description}

EAL7 elérése esetén olyan TOE-t kapunk, amely képes nagyértékű információ védelmére magas támadási
kockázattal rendelkező környezetben is biztonsággal működni. Gyakorlatban EAL7-et az érhet el,
amelynek a helyes működése formális eszközökkel is bizonyítható.
Emellett úgy növeli még az EAL6-hoz képest a bizonyosságot, hogy ezen a szinten meg kell osztani
a teljes implementációs reprezentációt (forráskódot) is.

