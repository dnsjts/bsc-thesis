\chapter{Háttérinformációk}

\section{Különböző megközelítések vizsgálata}

Az egyes megközelítések feltérképezéséhez és megvizsgálásához remek kiindulási alapot adott az
alábbi publikáció: \url{http://citeseerx.ist.psu.edu/viewdoc/download?doi=10.1.1.132.4843&rep=rep1&type=pdf}.

\subsection{Attack trees}

Az \emph{Attack trees} egy olyan módszer, amellyel egy adott támadást elemezhetünk egyszerűen, egy
faként reprezentálva az alábbi szabályok mentén:
\begin{enumerate}
    \item A fa gyökerében az az esemény áll, amely bekövetkezésekor a támadást sikeresnek tekintjük.
    \item A szülő és a gyerek eseménye között az a kapcsolat, hogy a szülő eseménye akkor, és 
        csak akkor következik be, ha a gyerek eseménye is.
        Összekapcsolt gyerekek esetén akkor és csak akkor, ha mindegyik gyerek eseménye bekövetkezett.
\end{enumerate}
Miután megtaláltuk az eseményeink függőségeit (pontosabban a további felbontásnak már nem lenne
értelme), lehetőségünk van ezeket különféle szempontokból elemezni, például az alábbi szempontok mentén:
\begin{itemize}
    \item Lehetséges-e egyáltalán az esemény bekövetkezése
    \item Mekkora költséggel jár előidézni az eseményt
    \item Mekkora költséggel jár megelőzni az eseményt
    \item Szükséges-e különleges felszerelés az előidézéshez
\end{itemize}

Az alábbi szempontok segíthetnek eldönteni, hogy megéri-e foglalkozni a védelem kialakításával,
szem előtt tartva, hogy mit feltételezünk a támadóról.

Habár a módszer kellően könnyedén értelmezhető (hiszen csupán egy fát használ) és könnyedén alkalmazható,
számunkra kevésbé használható, hiszen nem kellően széleskörű, és megoldásokat egyáltalán nem szolgáltat,
és a teljesség sem garantált.

\subsection{Abuse/Misuse cases}

Az \emph{Misuse cases} az ismert \emph{use case} diagramnak a kifordított változata, azaz ahelyett,
hogy azt modelleznénk, hogy egy adott szereplő milyen tevékenységeket végezhet el, itt inkább azt
modellezzük, hogy egy rosszindulatú szereplő milyen nemkívánt tevékenységeket végezne el, azaz
meghatározhatjuk azokat a tevékenységeket, amelyeket a \emph{rendszernek tiltania kellene}.

\subsection{ISO 27000-as család}
\subsubsection{ISO 27001:2013}
Ellentétben a korábbi megközelítésekkel, az \emph{ISO 27001}-es szabvány 

Habár elsőre nem tűnik indokoltnak a megemlítése, elég csupán arra gondolnunk, hogy
egy termék fejlesztésekor elvárhatjuk, hogy a belső infrastruktúránk biztonságos.
Ahogy a cég (és ezzel együtt az infrastruktúra) növekszik, egyre inkább fontossá
válik egy olyan kész keret bevezetése, amelyet felhasználva biztonságosabb rendszert építhetünk.

\subsection{Common Criteria}

\section{Miért választottuk a Common Criteria-t}

Végeredményként az alábbi metodologiákat kezdtük el alkalmazni:
\begin{itemize}
\item Common Criteria-t a 
\item{ISO 27001:2013-at az infrastruktúra (beleértve a számunkra releváns fejlesztői infrastruktúra)
    biztonságának erősítésére.}
\end{itemize}

A fő okok az alábbiak voltak:



\section{Common Criteria-ról bővebben}

\subsection{Terminológia}

\subsection{Evaluation Assurance Levels}

Az \emph{Evaluation Assurance Level} (továbbiakban: EAL) egy egytől hétig tartó skálán egyre nagyobb
bizonyosságot nyújt, pontosabban a magasabb szinteknél szigorúbb elvárásoknak kell megfelelni.
Természetesen a szigorúbb elvárásoknak való megfelelőség magasabb költséget is jelent,
így 

\subsubsection{EAL1: funkcionálisan tesztelt}
\begin{itemize}
    \item{Célja}
    \item{Miben több, mint az EAL0?}
\end{itemize}

\subsubsection{EAL2: strukturálisan tesztelt}
\begin{itemize}
    \item{Célja}
    \item{Miben több, mint az EAL1?}
\end{itemize}
\subsubsection{EAL3: metodilikusan tesztelt és ellenőrzött}
\begin{itemize}
    \item{Célja}
    \item{Miben több, mint az EAL2?}
\end{itemize}
\subsubsection{EAL4: }
\begin{itemize}
    \item{Célja}
    \item{Miben több, mint az EAL3?}
\end{itemize}
\subsubsection{EAL5:}
\begin{itemize}
    \item{Célja}
    \item{Miben több, mint az EAL4?}
\end{itemize}
\subsubsection{EAL6:}
\begin{itemize}
    \item{Célja}
    \item{Miben több, mint az EAL5?}
\end{itemize}
\subsubsection{EAL7:}
\begin{itemize}
    \item{Célja}
    \item{Miben több, mint az EAL6?}
\end{itemize}
