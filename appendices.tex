%----------------------------------------------------------------------------
\appendix
%----------------------------------------------------------------------------
\chapter*{Függelék}\addcontentsline{toc}{chapter}{Függelék}
\setcounter{chapter}{6}  % a fofejezet-szamlalo az angol ABC 6. betuje (F) lesz
\setcounter{equation}{0} % a fofejezet-szamlalo az angol ABC 6. betuje (F) lesz
\numberwithin{equation}{section}
\numberwithin{figure}{section}
\numberwithin{lstlisting}{section}
%\numberwithin{tabular}{section}

\section{Változtatások alkalmazása csak az újonnan beérkezett kódra}

\begin{lstlisting}[language=python]
#!/usr/bin/env python3

import subprocess

subprocess.run("find . -name '*.c' | xargs -n 1 clang-format -i && git commit --all --message Initial change", shell=True);
exit(0);
\end{lstlisting}

\begin{lstlisting}[language=bash]
#!/bin/bash

env SHELL="/bin/bash" GIT_SEQUENCE_EDITOR="sed -i '1s/^/exec .\\/initial.py\n/'" git rebase \
    --interactive \
    -X theirs \
    --exec 'git diff HEAD^ HEAD --name-only | grep -P ".c$" | xargs -n 1 clang-format -i && git commit --all --fixup HEAD --no-edit || echo "Nothing to do."' \
    $(git merge-base $1 HEAD);

env SHELL="/bin/bash" GIT_SEQUENCE_EDITOR="sed -i '1s/pick/drop/'" git rebase \
    --autosquash \
    --interactive \
    -X patience \
    -X theirs \
    $(git merge-base $1 HEAD);
\end{lstlisting}

A fenti kódrészlet annak az automatizmusnak a prototípusa, amely lehetővé teszi csak egy adott
módosítás automatikus formázását (azaz jelenleg ez a transzformáció bele van égetve).
Mint ahogy az látható, igazából ez csupán kettő interaktív \emph{git rebase}, amelyeknél megadtuk
a \emph{merge} stratégiákat annak érdekében, hogy ne kelljen kézzel feloldani az esetleges
konfliktusokat, és minden egyes \emph{commit} alkalmazása után lefuttatjuk a transzformációt,
továbbá az első commit előtt a teljes kódbázisra is.
A második \emph{rebase} a teljes kódbázisra futtatott transzformációt vonja vissza.

Sajnos a minimális különbség előállítása még nem sikerült, ugyanis a második rebase nem vonja vissza
teljesen a legelső transzformációt, így az eszköz alkalmazása felesleges módosításokat alkalmaz
az új kódon. Ennek megoldása valószínűleg a felváltva használandó merge stratégiákban keresendő.
