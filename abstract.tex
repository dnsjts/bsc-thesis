%----------------------------------------------------------------------------
% Abstract in hungarian
%----------------------------------------------------------------------------
\chapter*{Kivonat}\addcontentsline{toc}{chapter}{Kivonat}

Jelen szakdolgozat egy esettanulmány arról, hogy a \emph{syslog-ng} fejlesztése során hogyan
kíséreltük meg szisztematikusan felszámolni a felhalmozott technológiai adósságunkat a Common
Criteria követelményeinek megvalósításával, továbbá az így kapott erősebb minőségbiztosítás hogyan
fejleszti a termék biztonságát és egyszerűsíti a fejlesztést az automatizmusok révén.

Mivel a teljes Common Criteria teljes bevezetése rengeteg időt venne igénybe, és ennek bemutatása
terjedelmes lenne, ezért a szakdolgozatomban a termék fejlesztési és karbantartási folyamatára
alkalmazom ezeket az elveket, a terméktámogatástól a belső munkafolyamatokon át, a termék
kiadásáig.

\vfill

%----------------------------------------------------------------------------
% Abstract in english
%----------------------------------------------------------------------------
\chapter*{Abstract}\addcontentsline{toc}{chapter}{Abstract}

This document is a \LaTeX-based skeleton for BSc/MSc~theses of students at the Electrical
Engineering and Informatics Faculty, Budapest University of Technology and Economics. The usage of
this skeleton is optional. It has been tested with the \emph{TeXLive} \TeX~implementation, and it
requires the PDF-\LaTeX~compiler.
\vfill

