%----------------------------------------------------------------------------
% Abstract in hungarian
%----------------------------------------------------------------------------
\chapter*{Kivonat}\addcontentsline{toc}{chapter}{Kivonat}

Jelen szakdolgozat egy esettanulmány arról, hogy a \emph{syslog-ng} fejlesztése során hogyan
kíséreltük meg szisztematikusan felszámolni a felhalmozott technológiai adósságunkat a Common
Criteria követelményeinek megvalósításával, továbbá az így kapott erősebb minőségbiztosítás hogyan
fejleszti a termék biztonságát és egyszerűsíti a fejlesztést az automatizmusok révén.

Mivel a Common Criteria teljeskörű bevezetése rengeteg időt venne igénybe, és ennek bemutatása
terjedelmes lenne, ezért a szakdolgozatomban a termék fejlesztési és karbantartási folyamatára
alkalmazom ezeket az elveket, a terméktámogatástól a belső munkafolyamatokon át, a termék
kiadásáig.

\vfill

%----------------------------------------------------------------------------
% Abstract in english
%----------------------------------------------------------------------------
\chapter*{Abstract}\addcontentsline{toc}{chapter}{Abstract}

This thesis is a case study about how we attempted to systematically eliminate our accumulated
technogical debt by implementing the requirements of Common Criteria, and how the hereby obtained
stronger quality assurance helps to develop the security of the product and simplifies the
development through automatism.

Since it would take a lot of time to introduce the whole Common Criteria, and presenting this would
take an extensive amount of time, therefore in this thesis I apply these principles on the
development and maintenance of the product, from the product support, through the internal workflow,
to the release of the product.

\vfill

